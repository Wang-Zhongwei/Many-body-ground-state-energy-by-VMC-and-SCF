\documentclass[11pt]{article}
\usepackage{amsmath, amssymb, amscd, amsthm, amsfonts}
\usepackage{graphicx}
\usepackage{hyperref}
\usepackage{biblatex}
\addbibresource{references.bib}

\oddsidemargin 0pt
\evensidemargin 0pt
\marginparwidth 40pt
\marginparsep 10pt
\topmargin -20pt
\headsep 10pt
\textheight 8.7in
\textwidth 6.65in
\linespread{1.2}


\title{Numerical Methods for Solving Fermion System Ground State}
\author{Zhongwei Wang}
\date{2022 Autumn}

\newtheorem{theorem}{Theorem}
\newtheorem{lemma}[theorem]{Lemma}
\newtheorem{conjecture}[theorem]{Conjecture}

\newcommand{\rr}{\mathbb{R}}

\newcommand{\al}{\alpha}
\DeclareMathOperator{\conv}{conv}
\DeclareMathOperator{\aff}{aff}

\begin{document}

\maketitle

\begin{abstract}\footnote[0]{Source code available on \href{https://github.com/Wang-Zhongwei/Many-body-ground-state-energy-by-VMC-and-SCF/tree/main}{Github}}
This article presents numerical methods in solving the ground state of Fermion system, including self-consistent field (SCF) theory and variational monte carlo (VMC). 
As a demonstration of the algorithms, Helium and Beryllium atoms are solved independently by each algorithm in succession. 
Then a combination of the two methods is used to produce better ground state for Beryllium atom. 
It shows that each method alone produces results within 3\% of the exact value, while the combination of the two methods reduces the error to 0.1\% at least for the systems chosen in the article. (He and Be atoms)
\end{abstract}

\section{Introduction}\label{section-introduction}
Hamiltonians for a Fermion system has a general expression as follows:
\begin{equation}\label{eq:hamiltonian}
\hat{H} = \frac{1}{2}\sum_{i}\nabla_i^2  - \sum_i \frac{Z}{r_i} + \frac{1}{2}\sum_{i \neq j} \frac{1}{|r_i-r_j|},
\end{equation}
\begin{itemize}
    \item first two terms are called single body term, representing electron-nucleus interaction;
    \item last term called many body term, representing electron-electron interaction, which can be then break down into 
    direct term $J$ and exchange term $K$; 
\end{itemize}
Without the last term, system eigen states would be simply produced by Slater determinants. Otherwise, the system is not solvable analytically.

Common numerical methods for solving many body quantum systems include Variational Monte Carlo (VMC) \cite{First_VMC} methods, density functional theory (DFT) \cite{First-DFT}, and Self-consistent field (SCF) \cite{Early-SCF} thoery. 
These methods use various computational techniques to approximate the solutions to many body quantum systems. This article, in particular, uses self-consistent field (SCF) theory and Variational Monte Carlo (VMC) to solve the ground state of Helium and Beryllium. 
There are several mature libraries and software packages for quantum Monte Carlo and density functional theory simulation. For example, \href{https://www.quantum-espresso.org/}{Quantum ESPRESSO}, \href{https://qmcpack.org/}{QMCPACK}, \href{https://vallico.net/casinoqmc/}{CASINO} for VMC; \href{https://www.nwchem-sw.org/}{NWChem} and \href{https://dftbplus.org/}{DFTB+} for DFT. 

But for the purpose of illumination only, in this article I opt to implement them ground up. For the next two chapters I am going to review the principles of the two algorithms and lay out the procedure to solve the specific problems in conern, which is ground state of Helium and Beryllium.

\section{Principle}
In this chapter, derivation of the whole theory is not the main focus. One can refer to \cite{Derivation_of_Fock_matrix} \cite{Review_of_VMC} for detailed derivations. Instead, we review important conclusions about and stipulate notations for SCF and VMC.
\subsection{Hartree-Fock equations}
The Hartree-Fock self-consistent field (SCF) theory is a method used in quantum chemistry to approximate the wave function and energy of a many-electron system. 
The theory is based on the Hartree-Fock approximation, which assumes that the many-electron wave function can be represented as a determinant, known as Slater determinant, of single-electron wave functions (often not orthogonal), or orbitals. 
The SCF method involves iteratively solving the Hartree-Fock equations to determine the orbitals and the energy of the system, and then using these solutions as the starting point for the next iteration. 
This process is repeated until the solutions converge to a self-consistent set of orbitals and energy, which represents the best approximation of the true wave function and energy of the system.

Adapted from original Hamiltonian \ref{eq:hamiltonian}(How?). Hartree-Fock equation for a closed-shell atomic system reads as follows:
\cite{Fock_matrix_source}
\begin{equation}
\label{eq:hartree-fock}
\hat{f}(r_i) = \hat{h}(r_i)+\sum_{a=1}^{N/2}(2\hat{J}_a(r_i)-\hat{K}_a(r_i))
\end{equation}
where
\begin{itemize}
    \item $\hat{f}(r_i)$: Fock operator;
    \item $\hat{h}(r_i)$: single body operator;
    \item $\hat{J}_a(r_i)$: direct operator attributed to electron-electron coulomb repulsion;
    \item $\hat{K}_a(r_i)$: exchange operator attributed to same-spin electron exchange effect. The factor is half of that of the direct operator. 
\end{itemize}
One strong assumption of SCF is that the composite wavefunction can be written as the determinant of basis functions evaluated at each electron position. 

\begin{align*}
    \Psi(r_1, r_2, ..., r_N) &= 
    \begin{vmatrix}
    \psi_1(r_1) & \psi_2(r_1)  & \cdots & \psi_N(r_1) \\ 
    \psi_1(r_2) & \psi_2(r_2) & \cdots & \psi_N(r_2) \\ 
    \vdots & \vdots& \ddots & \vdots \\
    \psi_1(r_N) & \psi_2(r_N) & \cdots & \psi_N(r_N) \\ 
    \end{vmatrix} \\
\end{align*}

Remeber set of basis $\{\psi_1, ... , \psi_m\}$ is not neccessarily orthogonal. The Fock operator under the representation becomes:
\begin{equation}\label{eq:fock-representation}
    F_{\mu\nu}=H^{core}_{\mu\nu} + G_{\mu\nu}
\end{equation}
where 
\begin{itemize}
    \item $H^{core}_{\mu \nu} = \int{dv1 \phi_{\mu}^*(r_1)\hat{h}_1(r_1)\phi_{\nu}(r_1)}$; $\hat{h}_1(r_1)=-\frac{1}{2}\nabla^2_1-\frac{Z}{r_1}$; single body term.
    \item $G_{\mu\nu} = \sum_{a=1}^{N/2}(2(\mu\nu|aa)-(\mu a|a\nu))$
    \item $(\mu\nu|aa)=\int{dv_1 dv_2 \psi_{\mu}^*(r_1)\psi_{\nu}(r_1) \frac{1}{r_{12}} \psi_a^*(r_2)\psi_a(r_2)}$; direct term.
    \item $(\mu a|a\nu)=\int{dv_1 dv_2 \psi_{\mu}^*(r_1)\psi_{a}(r_1) \frac{1}{r_{12}} \psi_a^*(r_2)\psi_{\nu}(r_2)}$; exchange term.
\end{itemize}

The relation between energy and Fock operator is:
\begin{equation}
    E(\Psi(r_1, r_2, ..., r_N)) = \sum_{\mu, \nu} (H_{\mu\nu} + \frac{1}{2}G_{\mu\nu}) = \frac{1}{2} \sum_{\mu, \nu} (H_{\mu\nu} + F_{\mu\nu})
\end{equation}

However, it is more convient to choose another set of basis $\{\phi_1, \phi_2, ... \phi_N\}$ with relation $\psi_j = \sum_{i}\phi_{i} \cdot C_{ij}$, where $C_{ij}$ is called the coefficient matrix. 

\subsection{Variational Monte Carlo}


\section{Procedure}

For simplicity's sake, the system we only consider is a closed shell system where there are $N/2$ different spatial orbits given $N$ electrons. We assume form of the atomic wavefunction is 

\begin{equation}
\label{eq:atomic-wavefunction}
    \Psi = Det(\phi_1(r_1), \phi_2(r_2), ..., \phi_N(r_N)) \cdot \prod_{i<j}\exp(\frac{r_{ij}}{2(1 + \beta_{ij}r_{ij})})
\end{equation}
where the determinant part encapsulates electron exchange effect while the second part includes electron correlation which is not taken care of by Hartree-Fock approximation. 

For the basis function set, Slater type orbitals are used. 
\subsection{SCF}

\subsection{VMC}

\subsection{Combintation}


\section{Initialization}

\subsection{SCF}
\subsection{VMC}

\section{Results}

\section{Conclusion}

\printbibliography % see references.bib for bibliography management
\end{document}